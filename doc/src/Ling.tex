\section{Ling Αθροιστές}
\label{section:Ling}
Στα προηγούμενα κεφάλαια παρουσιάστηκαν διάφοροι δυαδικοί αθροιστές , 
ανάμεσα σε αυτούς και ο αθροιστής πρόβλεψης κρατουμένου. Έπειτα γνωστοποιήθηκαν 
διάφοροι τρόποι υπολογισμού των κρατουμένων για την υλοποίηση του CLA. Όπως
έχει αποδειχθεί οι δομές CLA είναι ιδανικές για την ελάττωση της καθυστέρησης υπολογισμού
του αποτελέσματος. Ωστόσο στο παρόν κεφάλαιο θα παρουσιαστεί μια βελτίωση που προτάθηκε 
από τον Ling \cite{5390638} .



\subsection{Βασική Θεωρία}

Αρχικά παρουσιάζεται η παρακάτω ισότητα 
\begin{equation}
    g_i = g_i*p_i
\end{equation}
\\
η οποία πηγάζει από 
\begin{equation*}
    a_i * b_i = a_i * b_i * (a_i + b_i)
\end{equation*}

Η βασική βελτιστοποίηση που επέφερε η θεωρία του Ling είναι η παρακάτω τροποποίηση 
της συνάρτησης υπολογισμού του σήματος Group Carry Generate του συνόλου $i$ έως $j$ με $i > j$
\begin{equation}
\begin{split}
    G_{i:j} &= g_i + p_ig_{i-1} + p_ip_{i-1}g_{i-2} + ... + p_ip_{i-1}p_{i-1}...p_{j+1}g_j\\
            &= p_ig_i + p_ig_{i-1} + p_ip_{i-1}g_{i-2} + ... + p_ip_{i-1}p_{i-1}...p_{j+1}g_j\\
            &= p_i \bigg[ g_i + g_{i-1} + p_{i-1}g_{i-2} + ... + p_{i-1}p_{i-1}...p_{j+1}g_j \bigg]
\end{split}
\end{equation}
Η τροποποίηση της συνάρτησης $G$ δημιουργεί έναν όρο μέσα στις αγκύλες ο οποίος
ονομάζεται $H$ κατά Ling και έχει τον παρακάτω ορισμό 
\begin{equation}
    H_{i:j} = g_i + G_{i-1:j}
\end{equation}
Επιπλέον για την ανάκτηση του σήματος $G$, δηλαδή του κρατούμενου που παράγει ένα σύνολο, 
που είναι και ο αρχικός σκοπός των CLA αθροιστών, γίνεται με την παρακάτω συνάρτηση :
\begin{equation}
    G_{i:j} = p_i * H_{i:j}
\end{equation}
Η υποστήριξη του τελεστή $\circledast$ καθώς και η αναγωγή σε πρόβλημα προθέματος αποτελεί μεγάλο πλεονέκτημα της παραγοντοποίησης αυτής. Έχοντας το σύνολο $i$ έως $j$ με $i>k>j$ αποδεικνύεται πως
\begin{equation}
\begin{split}
    G_{i:j} &= G_{i:k} + P_{i:k}*G_{k-1:j}\\
    p_i * H_{i:j}  &= p_i * H_{i:k} + P_{i:k}*\big( p_{k-1}*H_{k-1:j}\big)\\
            &= p_i \big[  H_{i:k} + P_{i-1:k-1}*H_{k-1:j}    \big] \\
    H_{i:j}  &= H_{i:k} + P_{i-1:k-1}*H_{k-1:j}
\end{split}
\end{equation}
\\
Ακολουθεί ένα απλό παράδειγμα για καλύτερη εμπέδωση αλλά και επαλήθευση 
\begin{equation*}
\begin{split}
    H_{7:2} &= H_{7:5} + P_{6:4}*H_{4:2} \\
    H_{7:5} &= g_7 + g_6 + p_6g_5 \\
    H_{4:2} &= g_4 + g_3 + p_3g_2 \\
    P_{6:4} &= p_6p_5p_4\\
    H_{7:2} &= g_7 + g_6 + p_6g_5 + p_6p_5p_4*\big(g_4 + g_3 + p_3g_2 \big)\\
            &= g_7 + g_6 + p_6g_5 + p_6p_5g_4 + p_6p_5p_4g_3 + p_6p_5p_4p_3g_2
\end{split}
\end{equation*}
Εφόσον στην παραγοντοποίηση που σύστησε ο Ling έχουν κληρονομηθεί όλα τα προνόμια 
των προθεματικών αθροιστών μένει η έκφραση του αποτελέσματος συναρτήσει του σήματος $H$.
Ως γνωστόν, κάθε ένα δυαδικό ψηφίο του αθροίσματος με την τεχνική πρόβλεψης κρατουμένου 
υπολογιζόταν με την συνάρτηση $sum_i = G_{i-1:0} \oplus x_i $ οπου $x_i = a_i \oplus b_i$.
Τελικά, το άθροισμα με την παραγοντοποίηση του Ling υπολογίζεται με :
\begin{equation}
\label{eq:ling_sum_1}
\equationame{άθροισμα κατά Ling}
\begin{split}
    sum_i &= G_{i-1:0} \oplus x_i \\
          &= (p_{i-1} * H_{i-1:0}) \oplus x_i
\end{split}
\end{equation}












\subsection{Πλεονεκτήματα της Ling παραγοντοποίησης}
Ένας αθροιστής Ling βελτιστοποιεί την επίδοση του CLA μειώνοντας κατά ένα το πλήθος 
εισόδων των λογικών πυλών (fan-in). Όμως αφαιρώντας από κάθε όρο του Group Generate 
σήματος ένα propagate έχει ως αποτέλεσμα την αύξηση της πολυπλοκότητας του τελευταίου 
σταδίου όπου υπολογίζεται το άθροισμα. 
Παρατηρείται στην εξίσωση \ref{eq:ling_sum_1} πως πρέπει να υπολογιστεί το $H$, 
στην συνέχεια να πραγματοποιηθεί η λογική του πράξη AND με το σήμα $p_i$ και τέλος
η έξοδος της πύλης AND να οδηγήσει την είσοδο της πύλης XOR σε συνδυασμό με το
σήμα $x_i$, όπως φαίνεται στην εικόνα \ref{fig:Ling_sum_1}.

\begin{figure}[H]
    \centering
    \begin{subfigure}{.4\textwidth}
        \centering
        \includegraphics[height=4.5cm,width=2.2cm]{Ling_sum_1.png}
        \caption{Απλή προσέγγιση}
        \label{fig:Ling_sum_1}
    \end{subfigure}
    \begin{subfigure}{.4\textwidth}
        \centering
        \includegraphics[height=4.5cm,width=4cm]{Ling_sum_2.png}
        \caption{Εναλλακτική προσέγγιση με πολυπλέκτη}
        \label{fig:Ling_sum_2}
    \end{subfigure}
    \caption{Λογική υπολογισμού του αθροίσματος κατά Ling}
    \label{fig:Ling_sum}
\end{figure}


% \\\\
% \textcolor{red}{[Figure with logic diagram of whats described above}
% \\\\
Η εξίσωση \ref{eq:ling_sum_1} μπορεί να εκφραστεί με μεγαλύτερη πολυπλοκότητα 
αλλά ταυτόχρονα μειώνοντας τον συνολικό χρόνο υπολογισμού του αθροίσματος.
\begin{equation}
\label{eq:ling_sum_2}
\equationame{άθροισμα κατά Ling (2)}
    sum_i = H_{i-1:0} ? \big(p_{i-1} \oplus x_i\big) : x_i
\end{equation}
% \\\\
% \textcolor{red}{[Figure with logic diagram of whats described above}
% \\\\
Στην τροποποίηση που εκφράζεται στην εξίσωση \ref{eq:ling_sum_2} και 
απεικονίζεται στην εικόνα \ref{fig:Ling_sum_2} με μία πρώτη ματιά φαίνεται να 
υπάρχει αρνητική απόδοση σε σχέση με την αρχική και πιο απλή υλοποίηση.
Στην πραγματικότητα όμως η απόδοση αφορά την καθυστέρηση και όχι το εμβαδόν. Το σήμα
$Η$ είναι αυτό που υπολογίζεται τελευταίο χρονικά και στην πρώτη αρχιτεκτονική
είναι στο κρίσιμο μονοπάτι. Αντίθετα στην δεύτερη υλοποίηση η λογική πύλη XOR 
έχει τελική τιμή στην έξοδο της πριν οριστικοποιηθεί η τιμή του σήματος H.

%------------------------------------------------
 Παρατηρείται, λοιπόν,  πως η μέθοδος του Ling μειώνει την πολυπλοκότητα μόνο στο πρώτο επίπεδο του αθροιστή, ενώ η αύξηση πολυπλοκότητας του τελευταίου επιπέδου δεν επιβαρύνει τον σχεδιασμό χρονικά σύμφωνα με την παραπάνω τροποποίηση.
 
 
 
 
 
 
 
 
\subsection{Αραίωση σε Ling Αρχιτεκτονικές}
Οι τεχνικές αραίωσης που παρουσιάστηκαν στην παράγραφο \ref{subsection:prefix_sparseness}
 είναι δυνατό να εφαρμοστούν και στην περίπτωση της παραγοντοποίησης κατά
 Ling \cite{1377160}.\\\\   
Sparsness-2
\begin{equation*}
    \begin{split}
        sum_i &= x_i \oplus G_{i-1:0}\\
              &= x_i \oplus p_{i-1}*H_{i-1:0}\\
              &= H_{i-1:0} ? x_i \oplus p_{i-1} : x_i\\
        sum_{i+1} &= x_{i+1} \oplus G_{i:0}\\
                  &= x_{i+1} \oplus (g_i + p_i*G_{i-1:0})\\
                  &= x_{i+1} \oplus (g_i + p_i*p_{i-1}*H_{i-1:0})\\
                  &= H_{i-1:0} ? x_{i+1} \oplus (g_i + p_i*p_{i-1}) : x_{i+1} \oplus g_i
    \end{split} 
\end{equation*}
Sparsness-4
\begin{equation*}
    \begin{split}
        sum_i =& x_i \oplus G_{i-1:0}\\
        =& x_i \oplus p_{i-1}*H_{i-1:0}\\
        =& H_{i-1:0} ? x_i \oplus p_{i-1} : x_i\\
        sum_{i+1} =& x_{i+1} \oplus G_{i:0}\\
        =& x_{i+1} \oplus (g_i + p_i*G_{i-1:0})\\
        =& x_{i+1} \oplus (g_i + p_i*p_{i-1}*H_{i-1:0})\\
        =& H_{i-1:0} ? x_{i+1} \oplus (g_i + p_i*p_{i-1}) : x_{i+1} \oplus g_i\\
        sum_{i+2} =& x_{i+2} \oplus G_{i+1:0}\\
        =& x_{i+2} \oplus (g_{i+1} + p_{i+1}g_i + p_{i+1}p_iG_{i-1:0})\\
        =& x_{i+2} \oplus (g_{i+1} + p_{i+1}g_i + p_{i+1}p_ip_{i-1}*H_{i-1:0})\\
        =& H_{i-1:0} ? x_{i+2} \oplus (g_{i+1} + p_{i+1}g_i + p_{i+1}p_ip_{i-1}) : x_{i+2} \oplus (g_{i+1} + p_{i+1}g_i)\\
        sum_{i+3} =& x_{i+3} \oplus G_{i+2:0}\\
        =& x_{i+3} \oplus (g_{i+2} + p_{i+2}g_{i+1} + p_{i+2}p_{i+1}g_i + p_{i+2}p_{i+1}p_iG_{i-1:0})\\
        =& x_{i+3} \oplus (g_{i+2} + p_{i+2}g_{i+1} + p_{i+2}p_{i+1}g_i + p_{i+2}p_{i+1}p_ip_{i-1}*H_{i-1:0})\\
        =& H_{i-1:0} ? x_{i+3} \oplus (g_{i+2} + p_{i+2}g_{i+1} + p_{i+2}p_{i+1}g_i + p_{i+2}p_{i+1}p_ip_{i-1}) \\&: x_{i+3} \oplus (g_{i+2} + p_{i+2}g_{i+1} + p_{i+2}p_{i+1}g_i)
    \end{split} 
\end{equation*}

 
 