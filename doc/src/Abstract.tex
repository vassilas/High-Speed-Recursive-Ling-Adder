\thispagestyle{empty}




\renewcommand{\abstractname}{Περίληψη}
\begin{abstract}
\normalsize
Η πρόσθεση είναι η βασικότερη αριθμητική πράξη, αποτελώντας τμήμα των περισσότερων 
ψηφιακών συστημάτων. Επίσης, είναι η πιο συχνή πράξη που εκτελείται από μία επεξεργαστική
μονάδα. Η επιτάχυνση της απασχολεί την επιστήμη των ψηφιακών κυκλωμάτων για αρκετά χρόνια.
Στα πλαίσια αυτής της εργασίας παρουσιάζεται μια νέα υλοποίηση αθροιστών υπολοίπου $2^n-1$. 
Τεχνικές και αρχιτεκτονικές που έχουν προταθεί στο παρελθόν, είτε αφορούν άμεσα τους αθροιστές 
υπολοίπου, είτε γενικότερα τους αθροιστές, συνδυάζονται για την ανάπτυξη του νέου σχεδιασμού.
Συγκεκριμένα και περιεκτικά, έγινε χρήση των μοντέλων κατά Jackson και Talwar, που αφορούν την γενίκευση
της παραγοντοποίησης που συστήθηκε από τον H. Ling. Εφαρμόζοντας αυτές τις λογικές συναρτήσεις
σε μία τοπολογία που πλεονεκτεί σε προθεματικά επίπεδα, επιτεύχθηκε και το τελικό αποτέλεσμα.
Οι αθροιστές που αναπτύχθηκαν συγκρίθηκαν με τις αποδοτικότερες τοπολογίες όσο άφορα την ταχύτητα,
το εμβαδόν και την κατανάλωση ενέργειας,
και τα αποτελέσματα είναι αρκετά υποσχόμενα. Ειδικότερα σε θεωρητικές προδιαγραφές 
η καθυστέρηση των νέων αθροιστών είναι αισθητά μικρότερη, ενώ σε πρακτικές συνθήκες 
οι αποδόσεις είναι θετικές σε ορισμένες περιπτώσεις και εξαρτώμενες της τεχνολογίας υλοποίησης. 
\end{abstract}



% \clearpage



% \renewcommand{\abstractname}{Αbstract}
% \begin{abstract}
% \normalsize
% \textcolor{red}{
% Addition is the fundamental arithmetic operation, constituting as part in most of digital systems. Furthermore is the most frequently used operation which executed from a processing unit.

% Η πρόσθεση είναι η βασικότερη αριθμητική πράξη, αποτελώντας τμήμα των περισσότερων 
% ψηφιακών συστημάτων. Επίσης, είναι η πιο συχνή πράξη που εκτελείται από μία επεξεργαστική
% μονάδα. Η επιτάχυνση της απασχολεί την επιστήμη των ψηφιακών κυκλωμάτων για αρκετά χρόνια.
% Στα πλαίσια αυτής της εργασίας παρουσιάζεται μια νέα υλοποίηση αθροιστών υπολοίπου $2^n-1$. 
% Τεχνικές και αρχιτεκτονικές που έχουν προταθεί στο παρελθόν, είτε αφορούν άμεσα τους αθροιστές 
% υπολοίπου, είτε γενικότερα τους αθροιστές, συνδυάζονται για την ανάπτυξη του νέου σχεδιασμού.
% Συγκεκριμένα και περιεκτικά, έγινε χρήση των μοντέλων κατά Jackson και Talwar, που αφορούν την γενίκευση
% της παραγοντοποίησης που συστήθηκε από τον H. Ling. Εφαρμόζοντας αυτές τις λογικές συναρτήσεις
% σε μία τοπολογία που πλεονεκτεί σε προθεματικά επίπεδα, επιτεύχθηκε και το τελικό αποτέλεσμα.
% Οι αθροιστές που αναπτύχθηκαν συγκρίθηκαν με τις αποδοτικότερες τοπολογίες όσο άφορα την ταχύτητα,
% το εμβαδόν και την κατανάλωση ενέργειας,
% και τα αποτελέσματα είναι αρκετά υποσχόμενα. Ειδικότερα σε θεωρητικές προδιαγραφές 
% η καθυστέρηση των νέων αθροιστών είναι αισθητά μικρότερη, ενώ σε πρακτικές συνθήκες 
% οι αποδόσεις είναι θετικές σε ορισμένες περιπτώσεις και εξαρτώμενες της τεχνολογίας υλοποίησης.
% }
% \end{abstract}






% Ο σχεδιασμός
% του αθροιστή αυτού βασίζεται σε τεχνικές και αρχιτεκτονικές που εχουν προταθεί στο παρελθόν
% είτε αφοροόυν τους αθροιστές υπολοίπου ειδικά, είτε τους αθροιστές γενικά.

% Το μεγαλύτερο χρονικό κόστος καταλαμβάνεται από τον υπολογισμό των κρατουμένων ανά ζεύγος ψηφίων. 
% Την δημοφιλέστερη δομή άθροισης αποτελούν οι αθροιστές πρόβλεψης κρατουμένου (Carry-Look-Ahead CLA).

% Με βάση την δομή των CLA αναπτύχθηκαν και οι προθεματική αθροιστές οι οποίοι σύστησαν ένα μεγάλο 
% σύνολο από αρχιτεκτονικές δέντρων με σκοπό την παραλληλοποίηση. 
% Αν και καταλαμβάνουν μεγαλύτερη ενέργεια και εμβαδόν συγκριτικά με τις απλές δομές, 
% έχουν αισθητά μικρότερη καθυστέρηση. Ο H. Ling, τπ 1981, πρότεινε μία παραγοντοποίηση
% η οποία είχε εφαρμογή στους προθεματικούς αθροιστές αθροιστές (Ling αθροιστές), 
% μειώνοντας την πολυπλοκότητα ενώ, το 2004, οι R. Jackson και S. Talwar γενίκευσαν την 
% παραγοντοποίηση αυτή προσφέροντας την δυνατότητα επιπλέον μείωσης της πολυπλοκότητας 
% υπολογισμού των κρατουμένων (Jackson αθροιστές).

% Στην παρούσα εργασία παρουσιάζεται ένας νέος σχεδιασμός αθροιστών υπολοίπου $2^n-1$.
% Με σκοπώ την επιτάχυνση των αθροιστών αυτών, συνδιάζεται η τεχνική παραγοντοποίησης 
% που προτάθηκε απο τον Jackson καθώς και η δομή που π

% Σκοπός της παρούσας εργασίας είναι η επιτάχυνση των αθροιστών υπολοίπου $2^n-1$ συνδυάζοντας
% την καλύτερη τοπολογία που έχει προταθεί και την παραγοντοποίηση που προσφέρουν οι αθροιστές Jackson.
% Αρχικά γίνεται μια συνοπτική αναφορά σε βασικές αρχιτεκτονικές πρόσθεσης και 
% αναλύονται οι αποδοτικότερες χρονικά τοπολογίες με επιπλέον αναφορά στις σχεδιαστικές δομές
% των αθροιστών υπολοίπου $2^n-1$. Αναλύονται σε βάθος οι παραγοντοποιήσεις Ling και Jackson 
% καθώς η ανάπτυξη των αθροιστών προς σύγκριση βασίζονται σε αυτές.
