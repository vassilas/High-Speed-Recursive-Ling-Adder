\section{Ανάπτυξη Αθροιστών υπολοίπου $2^n-1$ }

Σε αυτό το κεφάλαιο θα αναπτυχθούν συνολικά τριάντα-έξη αθροιστές υπολοίπου $2^n-1$
ακολουθώντας την αρχιτεκτονική που παρουσιάστηκε στο κεφάλαιο \ref{section:mod_addition}
με τα ελάχιστα επίπεδα. Ανάλογα με το είδος της παραγοντοποίησης που τους εφαρμόζεται 
οι αθροιστές ομαδοποιούνται σε τρεις ομάδες, απλοί Prefix, Ling και Jackson,
και σε κάθε ομάδα θα αναπτυχθεί ένας 8-bit, ένας 16-bit, ένας 32-bit και ένας 64-bit 
αθροιστής. 

Κάθε αθροιστής που αναπτύσσεται περιγράφεται πλήρως από τις λογικές 
συναρτήσεις κάθε επιπέδου του. Για κάθε αθροιστή επίσης εφαρμόζεται και η τακτική αραίωσης
sparseness-2 και sparseness-4 για την μείωση των πόρων που καταλαμβάνει. Για την περιγραφή των 
κυκλωμάτων χρησιμοποιήθηκε η γλώσσα περιγραφής υλικού Verilog \cite{1620780}, ενώ η προσομοίωση 
και ο έλεγχος ορθής περιγραφής και λειτουργίας υλοποιείται με την χρήση του εργαλείου VCS της 
Synopsys \cite{vcs-synopsys}. 






\subsection{Δομή και ανάπτυξη του κώδικα HDL}

Για την περιγραφή των κυκλωμάτων δημιουργήθηκε ένα αρχείο verilog, στο οποίο περιγράφονται
όλα τα δομικά στοιχεία από τα οποία δομούνται οι αθροιστές. Για κάθε δομικό στοιχείο-κόμβος
υλοποιήθηκε ξεχωριστό module στην verilog που περιγράφει την λογική του συνάρτηση. Για 
παράδειγμα, δημιουργήθηκε module για τον υπολογισμό των σημάτων propagate, generate και άθροισης
χωρίς κρατούμενο. Ο συγκεκριμένος κόμβος αποτελεί το πρώτο επίπεδο
όλων των αθροιστών, έχει δύο εισόδους, την a και b, και τρεις εξόδους. Στον πίνακά \ref{table:node_modules} του παραρτήματος 
\ref{appendix:Node_modules} δίνονται όλα τα modules που υλοποιήθηκαν μαζί με τις συναρτήσεις των εξόδων τους. Παρακάτω παρουσιάζεται η HDL περιγραφή των κόμβων του πρώτου επιπέδου όλων των αθροιστών, υπεύθυνοι για τον υπολογισμό των σημάτων propagate (p), generate (g) και ημι-αθροίσματος (x).


\begin{minted}
[
frame=lines,
framesep=2mm,
baselinestretch=1,
% bgcolor=LightGray,
fontsize=\footnotesize,
linenos
]{verilog}
module _gpx(a, b, g, p, x);

    input a, b;
    output g, p, x;
    
    assign p = a | b ;    // propagate 
    assign g = a & b ;    // generate
    assign x = a ^ b ;    // half-addition (witout carry in)

endmodule
\end{minted}

% Tropos grafis toy kwdika
% Επίσης, για θέματα βελτιστοποίησης επιλέχθηκε η περιγραφή των κυκλωμάτων να γίνει σε αρκετά χαμηλό επίπεδο. Κάθε κόμβος που αντιπροσωπεύει μια λογική συνάρτηση περιγράφηκε σε ξεχωριστή μονάδα. 

% Kwdikas Python - Code Generator
% Όμως στις περιπτώσεις όπου η είσοδος των αθροιστών είναι μεγάλη θα έπρεπε να γίνουν αρκετές κλήσεις των verilog-modules στον κώδικα. Για παράδειγμα, με μήκος εισόδου 64 δυαδικά ψηφία, μόνο και μόνο οι μονάδες που υπολογίζουν τα σήματα generate και propagate καλούνται 64 φορές και δεν έχει ακόμα αρχίσει η περιγραφή του δέντρου, πράγμα που καθιστά την γραφή του κώδικα δύσκολη. Για την αποφυγή ανθρώπινων λαθών, τα οποία θα καθυστερούσαν πολύ την διαδικασία του ελέγχου, αναπτύχθηκε κώδικας σε python για την εκτύπωση της Verilog περιγραφής του κάθε αθροιστή.

Είναι προφανές πως για την περιγραφή ενός αθροιστή το μόνο που απαιτείται είναι η χρήση των
κατάλληλων κόμβων, που αναφέρθηκαν στην προηγούμενη παράγραφο, με την σωστή σειρά και τις σωστές 
διασυνδέσεις μεταξύ των άλλων κόμβων. Άρα ο αθροιστής δεν απαιτεί επιπλέον λογική από αυτή που
έχει υλοποιηθεί ήδη, αρκεί να μην είναι ελλιπές το σύνολο των κόμβων που κατασκευάστηκε.
Όσον αφορά λοιπόν τους αθροιστές, η μόνη διαδικασία που απομένει για την ανάπτυξη τους είναι
η κλήση των modules-κόμβων και η ένωση τους με καλώδια. 

Για τους αθροιστές υπολοίπου αναπτύχθηκε λογισμικό σε python, υπεύθυνο για την αυτοματοποίηση της γραφής του κώδικα. Στην ουσία το πρόγραμμα αυτό δέχεται τα ονόματα των έτοιμων verilog-modules κόμβων και τα εκτυπώνει σε ένα αρχείο με την σωστή σειρά και τις διασυνδέσεις καλωδίων που απαιτούνται για την ορθή λειτουργία του αθροιστή. Την αρχιτεκτονική την ορίζει ο χρήστης. Η αναγκαιότητα του λογισμικού αυτού είναι αισθητή στην περίπτωση αθροιστών μεγάλου μήκους εισόδου, όπου κάθε κόμβος καλείται αρκετές φορές, με αποτέλεσμα η διαδικασία γραφής του κώδικα να είναι χρονοβόρα και ευάλωτη σε λάθη.



\subsection{Βασική δομή αθροιστών}

Στην ενότητα \ref{section:jackson} έγινε μια αναφορά σε σχεδιαστικούς κανόνες όσον αφορά τους 
αθροιστές Jackson. Για την ανάπτυξη ενός Jacson αθροιστή υπάρχει ένα μεγάλο πλήθος πιθανών υλοποιήσεων,
όχι μόνο στην επιλογή του προθεματικού δέντρου και το πλήθος των όρων της παραγοντοποίησης ( Radix-2, 
Radix-3, Radix-4 ... ), κάτι που αφορά και τους αθροιστές Prefix και Ling, αλλά 
και στην επιλογή της συνάρτησης παραγοντοποίησης σε κάθε επίπεδο.
% \\
% \textcolor{red}{[Γράψε όλους τους συνδυασμούς για έναν 8-bit ή 16-bit αθροιστή σε συντομία
% π.χ. για 8 = 2x2x2, 2x4, 4x2 , και σε κάθε ένα τις πιθανές Burgess Συναρτήσεις]}\\

Όσον αφορά την επιλογή του πλήθους εισόδων στους κόμβους του κάθε επίπεδου αποφεύγεται η επιλογή του Radix-2 διότι, αναδιατυπώνοντας, οι αντίστοιχες συναρτήσεις κατά Jacskon δεν διαφέρουν με αυτές ενός απλού Prefix. Θα ήταν αρκετά αποδοτική η υλοποίηση ορισμένων επιπέδων με Radix-3,  όπως και στους αθροιστές που αναπτύχθηκαν στα \cite{6189978} και \cite{6810474}, διότι στις περιπτώσεις περισσότερων των δύο εισόδων έχει εφαρμογή η παραγοντοποίηση και επιπλέον, εφόσον οι κόμβοι θα είναι των τριών εισόδων, αρά και οι συναρτήσεις των κόμβων αυτών πιο απλές, θα υπάρχουν περισσότερες αντιστοιχίες των έτοιμων υλοποιημένων λογικών συναρτήσεων από τις CMOS βιβλιοθήκες. Στην περίπτωση των αθροιστών $2^n-1$, με $n=8, 16, 32$ και 64, δεν υπάρχει πολλαπλάσιο του τρία που δίνει τουλάχιστον έναν από αυτούς τους αριθμούς. Στην πραγματικότητα υπάρχει τρόπος αλλά αφαιρείται το προνόμιο επαναχρησιμοποίησης κόμβων, με αποτέλεσμα η πολυπλοκότητα των αθροιστών καθώς και η επιφάνεια και η κατανάλωση να είναι αρκετά μεγαλύτερες. 

Σύμφωνα με αυτά που προ-ειπώθηκαν στον πίνακα \ref{tb:arch_2^n-1} παρουσιάζονται οι επιλογές που έγιναν, οι οποίες είναι κοινές για κάθε είδος αθροιστή ώστε τα αποτελέσματα των μετρήσεων να είναι έγκυρα.
\begin{table}[H]
\centering
     \begin{tabular}{ || c | c || } 
     \hline
     Μήκος εισόδου & Radix κάθε επιπέδου\\
     \hline\hline
     8-bit  & 2x4 \\
     16-bit & 4x4 \\
     32-bit & 2x4x4 \\
     64-bit & 4x4x4 \\
     \hline
     \end{tabular}
     \caption{Αρχιτεκτονική δομή των αθροιστών $2^n-1$ προς υλοποίηση}
     \label{tb:arch_2^n-1}
\end{table}

Στην εικόνα \ref{2^8-1_Tree_2x4} παρουσιάζεται η βασική δομή των αθροιστών υπολοίπου $2^8-1$ που θα χρησιμοποιηθεί παρακάτω. Ενώ η δομή είναι κοινή και για τα τρία είδη αθροιστών το κάθε σχήμα-κόμβος αντιπροσωπεύει διαφορετική λογική συνάρτηση. Επίσης, στο δέντρο αυτό ακολουθείται η κατασκευή μόνο των σημάτων που οδηγούν τον τελευταίο πολυπλέκτη ή το τελικό κρατούμενο, δηλαδή στην περίπτωση του Prefix το σήμα G, στου Ling το σήμα H και στο Jackson το R.
\begin{figure}[H]
    \centering
    \includegraphics[width=\textwidth]{J8_Color.png}
    \caption{Αρχιτεκτονική του $2^8-1$ αθροιστή υπολοίπου}
    \label{2^8-1_Tree_2x4}
\end{figure}

















\clearpage
\subsection{Διαδικασία ελέγχου ορθής λειτουργίας}
Για τον έλεγχο ορθής λειτουργίας των αθροιστών που αναπτύχθηκαν κατασκευάστηκε ένας αθροιστής 
υπολοίπου $2^n-1$ με την απλούστερη αρχιτεκτονική. Επίσης, υλοποιήθηκε μια διαδικασία ελέγχου,
η οποία τροφοδοτεί παράλληλα όλους τους αθροιστές με τυχαίους δυαδικούς αριθμούς και συγκρίνει 
την έξοδο τους με αυτήν των αθροιστών απλής δομής. Αυτός ο έλεγχος επαναλαμβάνεται αρκετές 
φορές, στην περίπτωση αυτή πραγματοποιήθηκαν 10000 επαναλήψεις, με τυχαίο σύνολο αριθμών 
κάθε φορά για ισχυρότερη πιστοποίηση της ισχυριζόμενης λειτουργίας.
\begin{figure}[H]
    \centering
    \includegraphics[width=\textwidth]{testbench.png}
    \caption{Test-Bench}
    \label{fig:testbench}
\end{figure}
Στην εικόνα \ref{fig:testbench} παρουσιάζεται η αρχιτεκτονική του συστήματος ελέγχου. Στην 
μονάδα σύγκρισης (Comparator) υπάρχει ένας δείκτης PASS αρχικοποιημένος με την τιμή μηδέν.
Σε κάθε επανάληψη αυξάνεται κατά μία μονάδα αν τα αποτελέσματα προς σύγκριση είναι ίδια.
Στο τέλος των επαναλήψεων το ποσοστό επιτυχίας είναι ίσο με 
\begin{equation*}
    SUCCESS = \frac{PASS}{\text{Number of Test-cycles}}
\end{equation*}
Όπως και σε κάθε περίπτωση σχεδιασμού υλικού, πρέπει ο έλεγχος σε επίπεδο προσομοίωσης να
έχει πλήρη επιτυχία, έτσι και σε αυτήν την περίπτωση ένας αθροιστής λειτουργεί σωστά όταν 
στο 100\% των πιθανών διανυσμάτων εισόδου η έξοδος είναι η αναμενόμενη σύμφωνα με τον σχεδιασμό.

Παρακάτω παρουσιάζεται η έξοδος του συστήματος προσομοίωσης για έλεγχο.


\begin{minted}
[
fontsize=\footnotesize,
framesep=2mm,
baselinestretch=1,
xleftmargin=0.06\textwidth
% bgcolor=LightGray,
]{text}
-------------------------------------------------------------------
 Length   | Architecture  | Rate                        | Results
-------------------------------------------------------------------
Prefix Adders :
-------------------------------------------------------------------
 8  - bit | 2x4           |       10000/      10000     | PASS
 16 - bit | 4x4           |       10000/      10000     | PASS
 32 - bit | 2x4x4         |       10000/      10000     | PASS
 64 - bit | 4x4x4         |       10000/      10000     | PASS
-------------------------------------------------------------------
Jackson Adders :
-------------------------------------------------------------------
 8  - bit | 2x4           |       10000/      10000     | PASS
 16 - bit | 4x4           |       10000/      10000     | PASS
 32 - bit | 2x4x4         |       10000/      10000     | PASS
 32 - bit | 4x4x2         |       10000/      10000     | PASS
 64 - bit | 4x4x4         |       10000/      10000     | PASS
-------------------------------------------------------------------
Ling Adders :
-------------------------------------------------------------------
 8  - bit | 2x4           |       10000/      10000     | PASS
 16 - bit | 4x4           |       10000/      10000     | PASS
 32 - bit | 2x4x4         |       10000/      10000     | PASS
 64 - bit | 4x4x4         |       10000/      10000     | PASS
-------------------------------------------------------------------
\end{minted}


  
    















\subsection{Αλγεβρική περιγραφή των αθροιστών}

Η περιγραφή των απλών Prefix καθώς και των Ling αθροιστών υπολοίπου $2^n-1$ 
εφαρμόζεται αλγεβρικά σε ένα πίνακα για το κάθε είδος παραγοντοποίησης εκτός
των αθροιστών Jackson όπου παρουσιάζεται μία πιο αναλυτική, αλγεβρική επίσης, προσέγγιση.
Στους παρακάτω πίνακες, στην δεξιά στήλη υποδεικνύεται το βάθος του επιπέδου στο 
προθεματικό δέντρο και συνοδεύεται από το πλήθος των εισόδων της παραγοντοποίησης 
που εφαρμόζεται στο επίπεδο αυτό (σε παρένθεση).

Με σκοπό την ευ-ανάγνωση των συναρτήσεων που θα διατυπωθούν παρακάτω, αναιρείται 
η παρακάτω συμβολική αναπαράσταση που δηλώθηκε σε προηγούμενα κεφάλαια.
\begin{equation*}
    X_i = X_{i:0}
\end{equation*}
όπου X ενα απο τα σήματα G,P,H,R και Q.
Με σκοπό την χρήση του υπογεγραμμένου ως δείκτη, δίνοντάς την παρακάτω νέα ερμηνεία
\begin{equation*}
    X_i = X_{i:i-k+1}
\end{equation*}
όπου ο όρος k συμβολίζει το πλήθος των εισόδων που συμπεριλαμβάνει το σήμα X. Ο όρος k
είναι ίσος με το γινόμενο των στοιχείων, στις παρενθέσεις, τις αριστερής στήλης, 
του επιπέδου αναφοράς και των προηγούμενων του. Για παράδειγμα στον πίνακα \ref{eq:Prefix_2^{n}-1}
το σήμα $G3_i$ του 64-bit αθροιστή, έχει $k = 4*4*4$, δηλαδή συμβολίζει το σήμα
$G3_{i:i-k+1} = G_{i:i-63} $.


\subsubsection{Prefix $2^n-1$}
% -----------------------------------------------------------------------------
\begin{table}[H]
\centering
     \begin{tabularx}{\textwidth}{ | g | X | } 

        \hline
        level & P-G Equations\\
        \hline
        \hline
 
        0   & 
        \begin{tabular}{@{}c@{}}
        $g_i = a_i * b_i$\\
        $p_i = a_i + b_i$\\
        $x_i = a_i \oplus b_i $
        \end{tabular}\\\hline
        
        % 8 - BIT
        % ----------------------------------------------------------------------
        \cellcolor{black} & \cellcolor{Gray}8-bit \\\hline
        
        1 (x2)  & 
        %\cellcolor{LightGray}
        \begin{tabular}{@{}c@{}}
        $G1_i = g_i + p_ig_{i-1}$\\
        $P1_i = p_i * p_{i-1}$
        \end{tabular}\\\hline

        2 (x4)  & 
        \begin{tabular}{@{}c@{}}
        $G2_i = G1_i + P1_{i}G1_{i-2} + P1_{i}P1_{i-2}G1_{i-4} +$ \\ $P1_{i}P1_{i-2}P1_{i-4}G1_{i-6}$
        \end{tabular}\\\hline
        
        % 16 - BIT
        % ----------------------------------------------------------------------
        \cellcolor{black} & \cellcolor{Gray}16-bit \\\hline
        
        1 (x4)  & 
        \begin{tabular}{@{}c@{}}
        $G1_i = g_i + p_ig_{i-1} + p_ip_{i-1}g_{i-1} + p_ip_{i-1}p_{i-2}g_{i-1}$\\
        $P1_i = p_i p_{i-1} p_{i-2} p_{i-3}$
        \end{tabular}\\\hline

        2 (x4)  & 
        \begin{tabular}{@{}c@{}}
        $G2_i = G1_i + P1_{i}G1_{i-4} + P1_{i}P1_{i-4}G1_{i-8} +$ \\ $P1_{i}P1_{i-4}P1_{i-8}G1_{i-12}$
        \end{tabular}\\\hline
        
        % 32 - BIT
        % ----------------------------------------------------------------------
        \cellcolor{black} & \cellcolor{Gray}32-bit \\\hline
        
        1 (x2)  & 
        \begin{tabular}{@{}c@{}}
        $G1_i = g_i + p_ig_{i-1}$\\
        $P1_i = p_i * p_{i-1}$
        \end{tabular}\\\hline

        2 (x4)  & 
        \begin{tabular}{@{}c@{}}
        $G2_i = G1_i + P1_{i}G1_{i-2} + P1_{i}P1_{i-2}G1_{i-4} +$ \\ $P1_{i}P1_{i-2}P1_{i-4}G1_{i-6}$\\
        $P2_i = P1_{i}P1_{i-2}P1_{i-4}P_{i-6}$
        \end{tabular}\\\hline
        
        3 (x4)  & 
        \begin{tabular}{@{}c@{}}
        $G3_i = G2_i + P2_{i}G2_{i-8} + P2_{i}P2_{i-8}G2_{i-16} +$ \\ $P2_{i}P2_{i-8}P2_{i-16}G2_{i-24}$\\
        \end{tabular}\\\hline
        
        % 64 - BIT
        % ----------------------------------------------------------------------
        \cellcolor{black} & \cellcolor{Gray}64-bit \\\hline
        
        1 (x4)  & 
        \begin{tabular}{@{}c@{}}
        $G1_i = g_i + p_ig_{i-1} + p_ip_{i-1}g_{i-1} + p_ip_{i-1}p_{i-2}g_{i-1}$\\
        $P1_i = p_i p_{i-1} p_{i-2} p_{i-3}$
        \end{tabular}\\\hline

        2 (x4)  & 
        \begin{tabular}{@{}c@{}}
        $G2_i = G1_i + P1_{i}G1_{i-4} + P1_{i}P1_{i-4}G1_{i-8} +$ \\ $P1_{i}P1_{i-4}P1_{i-8}G1_{i-12}$\\
        $P2_i = P1_{i}P1_{i-4}P1_{i-8}P1_{i-12}$
        \end{tabular}\\\hline
        
        3 (x4)  & 
        \begin{tabular}{@{}c@{}}
        $G3_i = G2_i + P2_{i}G2_{i-16} + P2_{i}P2_{i-16}G2_{i-32} +$ \\ $P2_{i}P2_{i-16}P2_{i-32}G2_{i-48}$\\
        \end{tabular}\\\hline
        

        \cellcolor{black} & \cellcolor{Gray} \\\hline
        
        SUM   & 
        \begin{tabular}{@{}c@{}}
        $ sum_i = G_{i-1} \oplus x_i$
        \end{tabular}\\\hline

    \end{tabularx}
\caption{Prefix $2^{n}-1$ Εξισώσεις}
\label{eq:Prefix_2^{n}-1}
\end{table}



%---------------------------------------------------
\subsubsection{Ling $2^n-1$}
%---------------------------------------------------
\begin{table}[H]
\centering
     \begin{tabularx}{\textwidth}{ | g | X | } 

        \hline
        level & H-P Equations\\
        \hline
        \hline
 
        0   & 
        \begin{tabular}{@{}c@{}}
        $g_i = a_i * b_i$\\
        $p_i = a_i + b_i$\\
        $x_i = a_i \oplus b_i $
        \end{tabular}\\\hline
        
        % 8 - BIT
        % ----------------------------------------------------------------------
        \cellcolor{black} & \cellcolor{Gray}8-bit \\\hline
        
        1 (x2)  & 
        %\cellcolor{LightGray}
        \begin{tabular}{@{}c@{}}
        $H1_i = g_i + g_{i-1}$\\
        $P1_i = p_i * p_{i-1}$
        \end{tabular}\\\hline

        2 (x4)  & 
        \begin{tabular}{@{}c@{}}
        $H2_i = H1_i + P1_{i-1}H1_{i-2} + P1_{i-1}P1_{i-3}H1_{i-4} +$ \\ $P1_{i-1}P1_{i-3}P1_{i-5}H1_{i-6}$
        \end{tabular}\\\hline
        
        % 16 - BIT
        % ----------------------------------------------------------------------
        \cellcolor{black} & \cellcolor{Gray}16-bit \\\hline
        
        1 (x4)  & 
        \begin{tabular}{@{}c@{}}
        $H1_i = g_i + g_{i-1} + p_{i-1}g_{i-2} + p_{i-1}p_{i-2}g_{i-3} $\\
        $P1_i = p_ip_{i-1}p_{i-2}p_{i-3}$
        \end{tabular}\\\hline

        2 (x4)  & 
        \begin{tabular}{@{}c@{}}
        $H2_i = H1_i + P1_{i-1}H1_{i-4} + P1_{i-1}P1_{i-5}H1_{i-8} +$ \\ $P1_{i-1}P1_{i-5}P1_{i-9}H1_{i-12}$
        \end{tabular}\\\hline
        
        % 32 - BIT
        % ----------------------------------------------------------------------
        \cellcolor{black} & \cellcolor{Gray}32-bit \\\hline
        
        1 (x2)  & 
        \begin{tabular}{@{}c@{}}
        $H1_i = g_i + g_{i-1}$\\
        $P1_i = p_i * p_{i-1}$
        \end{tabular}\\\hline

        2 (x4)  & 
        \begin{tabular}{@{}c@{}}
        $H2_i = H1_i + P1_{i-1}H1_{i-2} + P1_{i-1}P1_{i-3}H1_{i-4} +$ \\ $P1_{i-1}P1_{i-3}P1_{i-5}H1_{i-6}$\\
        $P2_i = P1_{i}P1_{i-2}P1_{i-4}P1_{i-6} $
        \end{tabular}\\\hline
        
        3 (x4)  & 
        \begin{tabular}{@{}c@{}}
        $H3_i = H2_i + P2_{i-1}H2_{i-8} + P2_{i-1}P2_{i-9}H2_{i-16} +$ \\ $P2_{i-1}P2_{i-9}P2_{i-17}H2_{i-24}$\\
        \end{tabular}\\\hline
        
        % 64 - BIT
        % ----------------------------------------------------------------------
        \cellcolor{black} & \cellcolor{Gray}64-bit \\\hline
        
        1 (x4)  & 
        \begin{tabular}{@{}c@{}}
        $H1_i = g_i + g_{i-1} + p_{i-1}g_{i-2} + p_{i-1}p_{i-2}g_{i-3} $\\
        $P1_i = p_ip_{i-1}p_{i-2}p_{i-3}$
        \end{tabular}\\\hline

        2 (x4)  & 
        \begin{tabular}{@{}c@{}}
        $H2_i = H1_i + P1_{i-1}H1_{i-4} + P1_{i-1}P1_{i-5}H1_{i-8} +$ \\ $P1_{i-1}P1_{i-5}P1_{i-9}H1_{i-12}$\\
        $P2_i = P1_{i}P1_{i-4}P1_{i-8}P1_{i-12}$
        \end{tabular}\\\hline
        
        3 (x4)  & 
        \begin{tabular}{@{}c@{}}
        $H3_i = H2_i + P2_{i-1}H2_{i-16} + P2_{i-1}P2_{i-17}H2_{i-32} +$ \\ $P2_{i-1}P2_{i-17}P2_{i-33}H2_{i-48}$\\
        \end{tabular}\\\hline
        

        \cellcolor{black} & \cellcolor{Gray} \\\hline
        
        SUM   & 
        \begin{tabular}{@{}c@{}}
        $ sum_i = H_{i-1}\ ?\ (x_i \oplus p_{i-1})\ :\ x_i$
        \end{tabular}\\\hline

    \end{tabularx}
\caption{Ling $2^{n}-1$ Εξισώσεις}
\end{table}





\clearpage
%---------------------------------------------------
\subsubsection{Jackson $2^n-1$}
%---------------------------------------------------

% -----------------------------------------------------------------------------
% 8-bit
% -----------------------------------------------------------------------------
Στους Jackson αθροιστές υπολοίπου $2^n-1$ η αλγεβρική περιγραφή γίνεται με αναλυτικό τρόπο
εφόσον αποτελούν το κύριο κομμάτι αυτής της αναφοράς. Συγκεκριμένα σε κάθε πίνακα 
εκτός από την συναρτησιακή περιγραφή δίνεται και η συμβολική ερμηνεία, έτσι ώστε
ο αναγνώστης να είναι σε θέση να αναπαράγει τους αθροιστές επικαλούμενος των
εξισώσεων της παραγράφου \ref{subsection:Jackson_Implementations}.

% -----------------------------------------------------------------------------
% 8-bit
% -----------------------------------------------------------------------------
\begin{table}[H]
\centering
     \begin{tabularx}{\textwidth}{ || g | X || } 

        \hline
        level & R-Q Equations\\
        \hline
        \hline
        
        0   & 
        \begin{tabular}{@{}c@{}}
        $g_i = a_i * b_i$\\
        $p_i = a_i + b_i$\\
        $x_i = a_i \oplus b_i $
        \end{tabular}\\\hline

        
        1 (x2)  & 
        \begin{tabular}{@{}c@{}}
        $R1_i = g_i + g_{i-1}$\\
        $Q1_i = p_i * p_{i-1}$
        \end{tabular}\\\hline

        2 (x4)  & 
        \begin{tabular}{@{}c@{}}
        $R2_i = R1_i + R1_{i-2} + Q1_{i-3}*R1_{i-4} + Q1_{i-3}*Q1_{i-5}*R1_{i-6}$
        \end{tabular}\\\hline

        D   & 
        \begin{tabular}{@{}c@{}}
        $ D_i = g_i + p_ig_{i-1} + p_ip_{i-1}p_{i-2}$
        \end{tabular}\\\hline

        SUM   & 
        \begin{tabular}{@{}c@{}}
        $ sum_i = R2_{i-1}\ ?\ (x_i \oplus D_{i-1})\ :\ x_i$
        \end{tabular}\\\hline

    \end{tabularx}
    
    
    \begin{tabularx}{\textwidth}{X} 
    \\
    \end{tabularx}    
    
    
    \begin{tabularx}{\textwidth}{| c | X X X X | } 
        \hline%\rowcolor{LightGray}
        Symbols & $R1_i$ & $Q1_i$ & $R2_i$ & $D_i$ \\
        \hline%\rowcolor{LightGray}
        Equations & $R^1_{i:i-1}$ & $Q^1_{i:i-1}$ & $R^3_{i:i-7}$ &$ D_{i:i-2}$ \\
        \hline
    \end{tabularx}
\caption{Jackson $2^{8}-1$ Εξισώσεις}
\end{table}

% -----------------------------------------------------------------------------
% 16-bit
% -----------------------------------------------------------------------------
\begin{table}[H]
\centering
     \begin{tabularx}{\textwidth}{|| g | X ||}
     
        \hline
        level & R-Q Equations\\
        \hline
        \hline
        0   & 
        \begin{tabular}{@{}c@{}}
        $g_i = a_i * b_i$\\
        $p_i = a_i + b_i$\\
        $x_i = a_i \oplus b_i $
        \end{tabular}\\\hline
        

        1 (x4)  & 
        \begin{tabular}{@{}c@{}}
        $R1_i = g_i + g_{i-1} + p_{i-1}g_{i-2} + p_{i-1}p_{i-2}g_{i-3}$\\
        $Q1_i = p_ip_{i-1}p_{i-2}p_{i-3}$
        \end{tabular}\\\hline
       

        2 (x4)  & 
        \begin{tabular}{@{}c@{}}
        $R2_i = R1_i + R1_{i-4} + Q1_{i-5}*R1_{i-8} + Q1_{i-5}*Q1_{i-9}*R1_{i-12}$
        \end{tabular}\\\hline
        

        D   & 
        \begin{tabular}{@{}c@{}}$ D_i = p_iR1_i + p_{i-1}Q1_i$
        \end{tabular}\\\hline
        

        SUM   & 
        \begin{tabular}{@{}c@{}}$ sum_i = R2_{i-1}\ ?\ (x_i \oplus D_{i-1})\ :\ x_i$
        \end{tabular}\\\hline
        
    \end{tabularx}
    
    
    \begin{tabularx}{\textwidth}{X} 
    \\
    \end{tabularx}  
    
    
    \begin{tabularx}{\textwidth}{| c | X X X X | } 
        \hline%\rowcolor{LightGray}
        Symbols & $R1_i$ & $Q1_i$ & $R2_i$ & $D_i$ \\
        \hline%\rowcolor{LightGray}
        Equations & $R^1_{i:i-3}$ & $Q^3_{i:i-3}$ & $R^5_{i:i-15}$ &$ D_{i:i-4}$ \\
        \hline
    \end{tabularx}
    
\caption{Jackson $2^{16}-1$ Εξισώσεις}
\end{table}

% -----------------------------------------------------------------------------
% 32-bit
% -----------------------------------------------------------------------------
\begin{table}[H]
\centering
     \begin{tabularx}{\textwidth}{|| g | X ||}
     
        \hline
        level & R-Q Equations\\
        \hline
        \hline
        0   & 
        \begin{tabular}{@{}c@{}}$g_i = a_i * b_i$\\$p_i = a_i + b_i$\\$x_i = a_i \oplus b_i $\end{tabular}\\\hline
        

        1 (x2)  & 
        \begin{tabular}{@{}c@{}}
        $R1_i = g_i + g_{i-1}$\\
        $Q1_i = p_i * p_{i-1}$
        \end{tabular}\\\hline

        2 (x4)  & 
        \begin{tabular}{@{}c@{}}
        $R2_i = R1_i + R1_{i-2} + Q1_{i-3}*R1_{i-4} + Q1_{i-3}*Q1_{i-5}*R1_{i-6}$\\
        $Q2_i = Q1_i Q1_{i-2} Q1_{i-4} ( R1_{i-5} + Q1_{i-6})$
        \end{tabular}\\\hline
        
        3 (x4)  & 
        \begin{tabular}{@{}c@{}}
        $R3_i = R2_i + R2_{i-8} + Q2_{i-11}*R2_{i-16} + Q2_{i-11}*Q2_{i-17}*R2_{i-24}$
        \end{tabular}\\\hline

        D   & 
        \begin{tabular}{@{}c@{}}$ D1_i = g_i + p_ig_{i-1} + p_ip_{i-1}p_{i-2}$\\
        $D_i = D1_i ( R2_i + Q2_{i-3} )$
        \end{tabular}\\\hline
        
        SUM   & 
        \begin{tabular}{@{}c@{}}$ sum_i = R3_{i-1}\ ?\ (x_i \oplus D_{i-1})\ :\ x_i$
        \end{tabular}\\\hline
        
    
    \end{tabularx}
    
    \begin{tabularx}{\textwidth}{X} 
    \\
    \end{tabularx}  
    
    
    \begin{tabularx}{\textwidth}{| c | X X X X X X | } 
        \hline%\rowcolor{LightGray}
        Symbols & $R1_i$ & $Q1_i$ & $R2_i$ & $Q2_i$ & $R3_i$ & $D_i$ \\
        \hline%\rowcolor{LightGray}
        Equations & $R^1_{i:i-1}$ & $Q^1_{i:i-1}$ & $R^3_{i:i-7}$ & $Q^5_{i:i-7}$ 
        & $R^{11}_{i:i-31}$ & $ D_{i:i-10}$ \\
        \hline
    \end{tabularx}
    
    
\caption{Jackson $2^{32}-1$ Εξισώσεις}
\end{table}

% -----------------------------------------------------------------------------
% 64-bit
% -----------------------------------------------------------------------------
\begin{table}[H]
\centering
     \begin{tabularx}{\textwidth}{|| g | X ||}
     
        \hline
        level & R-Q Equations\\
        \hline
        \hline
        
        0   & 
        \begin{tabular}{@{}c@{}}$g_i = a_i * b_i$\\$p_i = a_i + b_i$\\$x_i = a_i \oplus b_i $\end{tabular}\\\hline
        
        1 (x4)  & 
        \begin{tabular}{@{}c@{}}
        $R1_i = g_i + g_{i-1} + p_{i-1}g_{i-2} + p_{i-1}p_{i-2}g_{i-3}$\\
        $Q1_i = p_ip_{i-1}p_{i-2}p_{i-3}$
        \end{tabular}\\\hline
       
        2 (x4)  & 
        \begin{tabular}{@{}c@{}}
        $R2_i = R1_i + R1_{i-4} + Q1_{i-5}*R1_{i-8} + Q1_{i-5}*Q1_{i-9}*R1_{i-12}$\\
        $Q2_i = Q1_i Q1_{i-4} Q1_{i-8} ( R1_{i-11} + Q1_{i-12})$
        \end{tabular}\\\hline
        
        3 (x4)  & 
        \begin{tabular}{@{}c@{}}
        $R3_i = R2_i + R2_{i-16} + Q2_{i-21}*R2_{i-32} + Q2_{i-21}*Q2_{i-37}*R3_{i-48}$
        \end{tabular}\\\hline
        
        D   & 
        \begin{tabular}{@{}c@{}}$ D1_i = g_i + p_ig_{i-1} + p_ip_{i-1}p_{i-2}$\\
        $D2_i = D1_i ( R1_i + Q1_{i-3} )$\\
        $D_i = D2_i ( R2_i + Q2_{i-5} )$\\
        \end{tabular}\\\hline
        
        SUM   & 
        \begin{tabular}{@{}c@{}}$ sum_i = R3_{i-1}\ ?\ (x_i \oplus D_{i-1})\ :\ x_i$
        \end{tabular}\\\hline
    \end{tabularx}
    

    
    \begin{tabularx}{\textwidth}{X} 
    \\
    \end{tabularx}  
    
    
    \begin{tabularx}{\textwidth}{| c | X X X X X X | } 
        \hline%\rowcolor{LightGray}
        Symbols & $R1_i$ & $Q1_i$ & $R2_i$ & $Q2_i$ & $R3_i$ & $D_i$ \\
        \hline%\rowcolor{LightGray}
        Equations & $R^1_{i:i-3}$ & $Q^3_{i:i-3}$ & $R^5_{i:i-15}$ & $Q^{11}_{i:i-15}$ 
        & $R^{21}_{i:i-63}$ & $ D_{i:i-20}$ \\
        \hline
    \end{tabularx}
    
\caption{Jackson $2^{64}-1$ Εξισώσεις}
\end{table}

% -----------------------------------------------------------------------------
















%\subsubsection{$2^8-1$}
%---------------------------------------------------

% Figure
%
%--------------------------------------------


% \begin{table}[H]
% \centering
%     \begin{tabular}{|| c || c | c | c | c ||} 
%     \hline
%     Symbols & $R1_i$ & $Q1_i$ & $R2_i$ & $D_i$ \\
%     \hline
%     Equations & $R^1_{i:i-1}$ & $Q^1_{i:i-1}$ & $R^3_{i:i-7}$ &$ D_{i:i-2}$ \\
%     \hline
%     \end{tabular}
% \end{table}     


% Επίπεδο 1:\\
% \begin{equation}
% \begin{split}
% p_i &= a_i + b_i\\
% g_i &= a_i * b_i\\
% x_i &= a_i \oplus b_i
% \end{split}
% \end{equation}
% \\
% Επίπεδο 2:\\
% \begin{equation}
% \begin{split}
% R^1_{i:i-1} &= g_i + g_{i-1}\\
% Q^1_{i:i-1} &= p_i * p_{i-1}\\
% \end{split}
% \end{equation}
% \\
% Επίπεδο 3:\\
% \begin{equation}
% \begin{split}
% R^3_{i:i-7} =& R^1_{i:i-1} + R^1_{i-2:i-3} + Q^1_{i-3:i-4} R^1_{i-4:i-5} \\
%             +& Q^1_{i-3:i-4} Q^1_{i-5:i-6} R^1_{i-6:i-7} 
% \end{split}
% \end{equation}
% \\
% Group Generate:\\
% \begin{equation}
% G_{i:i-7} = D_{i:i-2} R^3_{i:i-7}
% \end{equation}
% Όπου : 
% \begin{equation}
% \begin{split}
% D_{i:i-2} &= G_{i:i-1} + P_{i:i-2}\\
% D_{i:i-2} &= g_i + p_ig_{i-1} + p_ip_{i-1}p_{i-2}
% \end{split}
% \end{equation}
% \\
% Επίπεδο 5 - Sum computation:\\
% \begin{equation}
% % sum_i = !R^3_{i-1:i-8} * (a_i \oplus b_i) + R^3_{i-1:i-8} * (a_i \oplus b_i \oplus D_{i-1:i-3})
% sum_i = R^3_{i-1:i-8} ? (x_i \oplus D_{i-1:i-3}) : x_i
% \end{equation}




% Για παράδειγμα:\\
% \rule{\linewidth}{0.5mm}
% \begin{equation*}
% \begin{split}
% p_7 =& a_7 + b_7\\
% g_7 =& a_7 * b_7\\
% x_7 =& a_7 \oplus b_7 \\
% R^1_{7:6} =& g_7 + g_{6}\\
% Q^1_{7:6} =& p_7 * p_{6}\\
% R^3_{7:0} =& R^1_{7:6} + R^1_{5:4} + Q^1_{4:3} R^1_{3:2} + Q^1_{4:3} Q^1_{2:1} R^1_{1:0}\\
% D_{7:5} =& g_7 + p_7g_{6} + p_7p_{6}p_{5}\\
% sum_7 =& R^3_{6:7}\ ?\ x_7 \oplus D_{6:4}\ :\ x_7 
% \end{split}
% \end{equation*}
% \rule{\linewidth}{0.5mm}



%\subsubsection{Λογική περιγραφή για $n={16, 32, 64}$}


%\subsubsection{$2^{16}-1$}
%%---------------------------------------------------
%Για παράδειγμα:\\
%\rule{\linewidth}{0.5mm}
%\begin{equation*}
%\begin{split}
%R^1_{15:12} =& g_{15} + g_{14} + p_{14}g_{13} + p_{14}p_{13}g_{12}\\
%Q^3_{15:12} =& p_{15} * p_{14} * p_{13} * p_{12}\\
%R^5_{15:0} =& R^1_{15:12} + R^1_{11:8} + Q^3_{10:7} R^1_{7:4} + Q^3_{10:7} Q^3_{6:3} R^1_{3:0}\\
%D_{15:11} =& p_{15}R^1_{15:12} + p_{11}Q^3_{15:12} \\
%sum_15 =& !R^5_{14:15} * (a_15 \oplus b_15) + R^5_{14:15} * (a_15 \oplus b_15 \oplus D_{14:10})
%\end{split}
%\end{equation*}
%\rule{\linewidth}{0.5mm}




%\subsubsection{$2^{32}-1$}
%%---------------------------------------------------
%
%
%
%
%
%\subsubsection{$2^{64}-1$}
%%---------------------------------------------------
%Για παράδειγμα:\\
%\rule{\linewidth}{0.5mm}
%\begin{equation*}
%\begin{split}
%R^1_{63:60} =& g_{63} + g_{62} + p_{62}g_{61} + p_{62}p_{61}g_{60}\\
%Q^3_{63:60} =& p_{63} * p_{62} * p_{61} * p_{60}\\
%R^5_{63:48} =& R^1_{63:60} + R^1_{59:56} + Q^3_{58:55} R^1_{55:52} + Q^3_{58:55} Q^3_{54:51} R^1_{51:48}\\
%Q^{11}_{63:48} =& Q^3_{63:60} Q^3_{59:56} Q^3_{55:52} ( R^1_{52:49} + Q^3_{51:48})\\
%R^{11}_{63:0} =& R^5_{63:48} + R^5_{47:32} + Q^{11}_{42:27} R^5_{31:16} + Q^{11}_{42:27} Q^{11}_{26:11} R^5_{15:0}\\
%D_{63:61} =& g_{63} + p_{63}g_{62} + p_{63}p_{62}p_{61}\\
%D_{63:59} =& D_{63:61} [R^1_{63:60} + Q^3_{62:59}]\\
%D_{63:43} =& D_{63:59} [R^5_{63:48} + Q^{11}_{58:43}]\\
%sum_63 =& !R^5_{62:63} * (a_63 \oplus b_63) + R^5_{62:63} * (a_63 \oplus b_63 \oplus D_{62:42})\\
%\end{split}
%\end{equation*}
%\rule{\linewidth}{0.5mm}












































% -----------------------------------------------------------------------------
% 8-bit
% -----------------------------------------------------------------------------
% \begin{table}[H]
% \centering
%      \begin{tabularx}{\textwidth}{ | g | X | } 

%         \hline
%         level & P-G Equations\\
%         \hline
%         \hline
 
%         0   & 
%         \begin{tabular}{@{}c@{}}
%         $g_i = a_i * b_i$\\
%         $p_i = a_i + b_i$\\
%         $x_i = a_i \oplus b_i $
%         \end{tabular}\\\hline

%         1 (x2)  & 
%         %\cellcolor{LightGray}
%         \begin{tabular}{@{}c@{}}
%         $G1_i = g_i + p_ig_{i-1}$\\
%         $P1_i = p_i * p_{i-1}$
%         \end{tabular}\\\hline

%         2 (x4)  & 
%         \begin{tabular}{@{}c@{}}
%         $G2_i = G1_i + P1_{i}G1_{i-2} + P1_{i}P1_{i-2}G1_{i-4} +$ \\ $P1_{i}P1_{i-2}P1_{i-4}G1_{i-6}$
%         \end{tabular}\\\hline

%         SUM   & 
%         \begin{tabular}{@{}c@{}}
%         $ sum_i = G_{i-1} \oplus x_i$
%         \end{tabular}\\\hline

%     \end{tabularx}
% \caption{Prefix $2^{8}-1$ Equations}
% \end{table}



% -----------------------------------------------------------------------------
% 16-bit
% -----------------------------------------------------------------------------
% \begin{table}[H]
% \centering
%      \begin{tabularx}{\textwidth}{ | g | X | } 

%         \hline
%         level & P-G Equations\\
%         \hline
%         \hline
 
%         0   & 
%         \begin{tabular}{@{}c@{}}
%         $g_i = a_i * b_i$\\
%         $p_i = a_i + b_i$\\
%         $x_i = a_i \oplus b_i $
%         \end{tabular}\\\hline

%         1 (x4)  & 
%         \begin{tabular}{@{}c@{}}
%         $G1_i = g_i + p_ig_{i-1} + p_ip_{i-1}g_{i-1} + p_ip_{i-1}p_{i-2}g_{i-1}$\\
%         $P1_i = p_i p_{i-1} p_{i-2} p_{i-3}$
%         \end{tabular}\\\hline

%         2 (x4)  & 
%         \begin{tabular}{@{}c@{}}
%         $G2_i = G1_i + P1_{i}G1_{i-4} + P1_{i}P1_{i-4}G1_{i-8} +$ \\ $P1_{i}P1_{i-4}P1_{i-8}G1_{i-12}$
%         \end{tabular}\\\hline

%         SUM   & 
%         \begin{tabular}{@{}c@{}}
%         $ sum_i = G_{i-1} \oplus x_i$
%         \end{tabular}\\\hline

%     \end{tabularx}
% \caption{Prefix $2^{16}-1$ Equations}
% \end{table}


% -----------------------------------------------------------------------------
% 32-bit
% -----------------------------------------------------------------------------
% \begin{table}[H]
% \centering
%      \begin{tabularx}{\textwidth}{ | g | X | } 
    
%         \hline
%         level & P-G Equations\\
%         \hline
%         \hline
 
%         0   & 
%         \begin{tabular}{@{}c@{}}
%         $g_i = a_i * b_i$\\
%         $p_i = a_i + b_i$\\
%         $x_i = a_i \oplus b_i $
%         \end{tabular}\\\hline

%         1 (x2)  & 
%         \begin{tabular}{@{}c@{}}
%         $G1_i = g_i + p_ig_{i-1}$\\
%         $P1_i = p_i * p_{i-1}$
%         \end{tabular}\\\hline

%         2 (x4)  & 
%         \begin{tabular}{@{}c@{}}
%         $G2_i = G1_i + P1_{i}G1_{i-2} + P1_{i}P1_{i-2}G1_{i-4} +$ \\ $P1_{i}P1_{i-2}P1_{i-4}G1_{i-6}$\\
%         $P2_i = P1_{i}P1_{i-2}P1_{i-4}P_{i-6}$
%         \end{tabular}\\\hline
        
%         3 (x4)  & 
%         \begin{tabular}{@{}c@{}}
%         $G3_i = G2_i + P2_{i}G2_{i-8} + P2_{i}P2_{i-8}G2_{i-16} +$ \\ $P2_{i}P2_{i-8}P2_{i-16}G2_{i-24}$\\
%         \end{tabular}\\\hline
        
%         SUM   & 
%         \begin{tabular}{@{}c@{}}
%         $ sum_i = G_{i-1} \oplus x_i$
%         \end{tabular}\\\hline

%     \end{tabularx}
% \caption{Prefix $2^{32}-1$ Equations}
% \end{table}


% -----------------------------------------------------------------------------
% 64-bit
% -----------------------------------------------------------------------------
% \begin{table}[H]
% \centering
%      \begin{tabularx}{\textwidth}{ || g | X || } 

%         \hline
%         level & P-G Equations\\
%         \hline
%         \hline
 
%         0   & 
%         \begin{tabular}{@{}c@{}}
%         $g_i = a_i * b_i$\\
%         $p_i = a_i + b_i$\\
%         $x_i = a_i \oplus b_i $
%         \end{tabular}\\\hline

%         1 (x4)  & 
%         \begin{tabular}{@{}c@{}}
%         $G1_i = g_i + p_ig_{i-1} + p_ip_{i-1}g_{i-1} + p_ip_{i-1}p_{i-2}g_{i-1}$\\
%         $P1_i = p_i p_{i-1} p_{i-2} p_{i-3}$
%         \end{tabular}\\\hline

%         2 (x4)  & 
%         \begin{tabular}{@{}c@{}}
%         $G2_i = G1_i + P1_{i}G1_{i-4} + P1_{i}P1_{i-4}G1_{i-8} +$ \\ $P1_{i}P1_{i-4}P1_{i-8}G1_{i-12}$\\
%         $P2_i = P1_{i}P1_{i-4}P1_{i-8}P1_{i-12}$
%         \end{tabular}\\\hline
        
%         3 (x4)  & 
%         \begin{tabular}{@{}c@{}}
%         $G3_i = G2_i + P2_{i}G2_{i-16} + P2_{i}P2_{i-16}G2_{i-32} +$ \\ $P2_{i}P2_{i-16}P2_{i-32}G2_{i-48}$\\
%         \end{tabular}\\\hline
        
%         SUM   & 
%         \begin{tabular}{@{}c@{}}
%         $ sum_i = G_{i-1} \oplus x_i$
%         \end{tabular}\\\hline

%     \end{tabularx}
% \caption{Prefix $2^{64}-1$ Equations}
% \end{table}



































% -----------------------------------------------------------------------------
% 8-bit
% -----------------------------------------------------------------------------
% \begin{table}[H]
% \centering
%      \begin{tabularx}{\textwidth}{ || g | X || } 

%         \hline
%         level & P-H Equations\\
%         \hline
%         \hline
        
%         0   & 
%         \begin{tabular}{@{}c@{}}
%         $g_i = a_i * b_i$\\
%         $p_i = a_i + b_i$\\
%         $x_i = a_i \oplus b_i $
%         \end{tabular}\\\hline

        
%         1 (x2)  & 
%         \begin{tabular}{@{}c@{}}
%         $H1_i = g_i + g_{i-1}$\\
%         $P1_i = p_i * p_{i-1}$
%         \end{tabular}\\\hline

%         2 (x4)  & 
%         \begin{tabular}{@{}c@{}}
%         $H2_i = H1_i + P1_{i-1}H1_{i-2} + P1_{i-1}P1_{i-3}H1_{i-4} +$ \\ $P1_{i-1}P1_{i-3}P1_{i-5}H1_{i-6}$
%         \end{tabular}\\\hline


%         SUM   & 
%         \begin{tabular}{@{}c@{}}
%         $ sum_i = H2_{i-1}\ ?\ (x_i \oplus p_{i-1})\ :\ x_i$
%         \end{tabular}\\\hline

%     \end{tabularx}
% \caption{Ling $2^{8}-1$ Equations}
% \end{table}


% -----------------------------------------------------------------------------
% 16-bit
% -----------------------------------------------------------------------------
% \begin{table}[H]
% \centering
%      \begin{tabularx}{\textwidth}{ || g | X || } 

%         \hline
%         level & P-H Equations\\
%         \hline
%         \hline
        
%         0   & 
%         \begin{tabular}{@{}c@{}}
%         $g_i = a_i * b_i$\\
%         $p_i = a_i + b_i$\\
%         $x_i = a_i \oplus b_i $
%         \end{tabular}\\\hline

        
%         1 (x4)  & 
%         \begin{tabular}{@{}c@{}}
%         $H1_i = g_i + g_{i-1} + p_{i-1}g_{i-2} + p_{i-1}p_{i-2}g_{i-3} $\\
%         $P1_i = p_ip_{i-1}p_{i-2}p_{i-3}$
%         \end{tabular}\\\hline

%         2 (x4)  & 
%         \begin{tabular}{@{}c@{}}
%         $H2_i = H1_i + P1_{i-1}H1_{i-4} + P1_{i-1}P1_{i-5}H1_{i-8} +$ \\ $P1_{i-1}P1_{i-5}P1_{i-9}H1_{i-12}$
%         \end{tabular}\\\hline


%         SUM   & 
%         \begin{tabular}{@{}c@{}}
%         $ sum_i = H2_{i-1}\ ?\ (x_i \oplus p_{i-1})\ :\ x_i$
%         \end{tabular}\\\hline

%     \end{tabularx}
% \caption{Ling $2^{16}-1$ Equations}
% \end{table}




% -----------------------------------------------------------------------------
% 32-bit
% -----------------------------------------------------------------------------
% \begin{table}[H]
% \centering
%      \begin{tabularx}{\textwidth}{ || g | X || } 

%         \hline
%         level & P-H Equations\\
%         \hline
%         \hline
        
%         0   & 
%         \begin{tabular}{@{}c@{}}
%         $g_i = a_i * b_i$\\
%         $p_i = a_i + b_i$\\
%         $x_i = a_i \oplus b_i $
%         \end{tabular}\\\hline

        
%         1 (x2)  & 
%         \begin{tabular}{@{}c@{}}
%         $H1_i = g_i + g_{i-1}$\\
%         $P1_i = p_i * p_{i-1}$
%         \end{tabular}\\\hline

%         2 (x4)  & 
%         \begin{tabular}{@{}c@{}}
%         $H2_i = H1_i + P1_{i-1}H1_{i-2} + P1_{i-1}P1_{i-3}H1_{i-4} +$ \\ $P1_{i-1}P1_{i-3}P1_{i-5}H1_{i-6}$\\
%         $P2_i = P1_{i}P1_{i-2}P1_{i-4}P1_{i-6} $
%         \end{tabular}\\\hline

        
%         3 (x4)  & 
%         \begin{tabular}{@{}c@{}}
%         $H3_i = H2_i + P2_{i-1}H2_{i-8} + P2_{i-1}P2_{i-9}H2_{i-16} +$ \\ $P2_{i-1}P2_{i-9}P2_{i-17}H2_{i-24}$\\
%         \end{tabular}\\\hline
        
%         SUM   & 
%         \begin{tabular}{@{}c@{}}
%         $ sum_i = H3_{i-1}\ ?\ (x_i \oplus p_{i-1})\ :\ x_i$
%         \end{tabular}\\\hline

%     \end{tabularx}
    
    

% \caption{Ling $2^{32}-1$ Equations}
% \end{table}




% -----------------------------------------------------------------------------
% 64-bit
% -----------------------------------------------------------------------------
% \begin{table}[H]
% \centering
%      \begin{tabularx}{\textwidth}{ || g | X || } 

%         \hline
%         level & P-H Equations\\
%         \hline
%         \hline
        
%         0   & 
%         \begin{tabular}{@{}c@{}}
%         $g_i = a_i * b_i$\\
%         $p_i = a_i + b_i$\\
%         $x_i = a_i \oplus b_i $
%         \end{tabular}\\\hline

        
%         1 (x4)  & 
%         \begin{tabular}{@{}c@{}}
%         $H1_i = g_i + g_{i-1} + p_{i-1}g_{i-2} + p_{i-1}p_{i-2}g_{i-3} $\\
%         $P1_i = p_ip_{i-1}p_{i-2}p_{i-3}$
%         \end{tabular}\\\hline

%         2 (x4)  & 
%         \begin{tabular}{@{}c@{}}
%         $H2_i = H1_i + P1_{i-1}H1_{i-4} + P1_{i-1}P1_{i-5}H1_{i-8} +$ \\ $P1_{i-1}P1_{i-5}P1_{i-9}H1_{i-12}$\\
%         $P2_i = P1_{i}P1_{i-4}P1_{i-8}P1_{i-12}$
%         \end{tabular}\\\hline

%         3 (x4)  & 
%         \begin{tabular}{@{}c@{}}
%         $H3_i = H2_i + P2_{i-1}H2_{i-16} + P2_{i-1}P2_{i-17}H2_{i-32} +$ \\ $P2_{i-1}P2_{i-17}P2_{i-33}H2_{i-48}$\\
%         \end{tabular}\\\hline
        
%         SUM   & 
%         \begin{tabular}{@{}c@{}}
%         $ sum_i = H2_{i-1}\ ?\ (x_i \oplus p_{i-1})\ :\ x_i$
%         \end{tabular}\\\hline

%     \end{tabularx}
% \caption{Ling $2^{64}-1$ Equations}
% \end{table}

%\subsubsection{$2^8-1$}
%%---------------------------------------------------
%\subsubsection{$2^{16}-1$}
%\subsubsection{$2^{32}-1$}
%\subsubsection{$2^{64}-1$}





